\documentclass[a4paper]{article}
\usepackage{preamble}
\usepackage{374_preamble}

\begin{document}

Rafal Weron, Modeling and Forecasting Electricity Loads and Prices: A Statistical Approach.

Refal Weron has written a book on statistical methods in electricity price forecasting. The opening chapter of this book explains the importance of having what he calls a price quotation mechanism, that is a price forecasting model. He explains why market liberalization must be coupled with a good price quotation mechanism, in order for the benefits of energy market liberalization to materialize. \\

\breathe
Rafal Weron, Electricity price forecasting: A review of the state-of-the-art with a look into the future.

Rafal has also written a review of different methods in use. I have not read this. It may not offer more than a wikipedia summary.

\breathe

Derek W. Bunn and Nektaria Karakatsani
Forecasting Electricity Prices 

This is a review, explaining the economic factors that influence spot pricing. For example, at times retailers need to boost their capacity -- for example, at times of high demand. At such times, retailers may need to start an otherwise inactive plant. The high operational costs of starting up this once dormant plant must be recovered -- and, so, spot prices rise. 

\begin{definition}
    A spot price is a price for a commodity decided upon today or ``now''. If the commodity is bought at the spot price, the settlement (the exchange of the commodity for the spot price) is done in the immediate future (e.g. after two days). This is in contrast to futures, which are settled far ahead of their transaction date.
\end{definition}
\end{document}



